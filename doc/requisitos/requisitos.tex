\documentclass[a4paper,12pt]{article}
\usepackage[utf8]{inputenc}
\usepackage[spanish]{babel}
\usepackage{graphicx}
\usepackage{epstopdf}
\usepackage{graphicx}
\usepackage{listings}
\usepackage{array}
\usepackage{hyperref}
\usepackage[margin=25mm]{geometry}

\title{Documento de Requisitos}
\author{Carlos Bergen Dyck \and Gustav Svensk}
\begin{document}

\renewcommand{\arraystretch}{1.5}
\maketitle
\begin{center}
        {\large Versión 0.1}
\end{center}
\newpage


\section{Prefacio}
\begin{tabular}{p{3cm} p{12cm}}
        & Este es el Documento de Requisitos de Air Guitar, que consiste en
        producir sonido cuando se toca a su Air Guitar. \\
        \textbf{Alcance del documento} & El Documento de Requisitos es la base
        de todo el desarrollo futuro de Nombre del Sistema. Describe los
        siguientes aspectos del sistema: propósito, contexto, requisitos
        funcionales, requisitos de pruebas, requisitos de calidad, requisitos
        de ambiente, arquitectura del sistema, requisitos del desarrollo,
        requisitos de post-desarrollo, y riesgos del proyecto. \\
        \textbf{Documentos relacionados} & Documento de Inicio de Proyecto,
        proyecto Nombre del Sistema, versión 1.1, 5/11/2013. \\
        \textbf{Autor} & Carlos Bergen Dyck y Gustav Svensk \\
        \textbf{Lectores} & Este documento está dirigido principalmente a los
        desarrolladores del proyecto, pero es de interés de todos los
        interesados en el mismo. \\
\end{tabular}

\section{Historia del Documento}
\begin{tabular}{|c|c|p{6cm}|p{4cm}|}
        \hline
        \textbf{Versión} & \textbf{Fecha} & \textbf{Explicación del cambio} &
        \textbf{Autor} \\ \hline
        0.1 & 30/10/2013 & Primer borrador & Carlos Bergen Dyck y Gustav Svensk \\
        \hline
\end{tabular}

\newpage
\tableofcontents

\listoftables

\listoffigures
\newpage

\section{Introducción}
En esta introducción se describe brevemente el contexto, objetivos y alcance
del proyecto a desarrollar, así como la documentación relativa al mismo. Esta
información está basada en el Documento de Inicio de Proyecto.

\subsection{Propósito}
Se generará un sonido de guitarra cuando un usuario esté simulando el tocar
una guitarra en el aire. El tono y el volumen se basaran en la posición de
las manos y la cadera del usuario.

\subsection{Alcance}
El sistema generará sonidos de acordes de guitarra. El tono se calculará en
base a la distancia entre la mano izquierda y la cadera. El volumen se
calculará en base a la velocidad de la mano derecha. Los sonidos se
reproducirán cuando la mano derecha pase sobre un área predefinida que
simulara el área de las cuerdas de una guitarra real.  El sistema funcionará
con un solo usuario a la vez. Deberá estar parado, de frente al Kinect y a
una distancia de entre 0.8m  a 2.5m aproximadamente debido a las limitaciones
técnicas del Kinect \cite{depth_range}. Ademas debe existir una iluminación adecuada. Solo se
podrán tocar power chords (acordes que consisten de la nota fundamental, su
quinta y octavas). El rango de sonidos será similar a una guitarra real
siendo la nota fundamental más grave E1 y la más aguda E2, así cubriendo una
octava, generando un total de 12 acordes posibles en todos los semitonos. 

\subsection{Contexto}
“Air Guitar” es un movimiento en donde se pretende tocar una guitarra.
Usualmente estos movimientos carecen de sonido. Desde 1996 se ha realizado
anualmente el Campeonato Mundial de Air Guitar en Finlandia \cite{air_guitar}.

Kinect es un sensor de movimiento creado inicialmente para el Xbox 360.
Utiliza una cámara RGB y un sensor infrarrojo para detectar la distancia de
diferentes objetos y seguir los movimientos del cuerpo humano \cite{kinect_spec}.

Microsoft libero el SDK de Kinect for Windows en el 2011 \cite{kinect_release}. Esto permitió a los
desarrolladores crear aplicaciones que usaran el Kinect en C++/CLI, C\# y
Visual Basic .NET. Este SDK permite reconocer movimientos, gestos y comandos
de voz \cite{sdk}. La ultima version también incluye herramientas para quitar el fondo
para obtener un efecto de pantalla verde, mejoras para escanear y modelar
objetos 3D con Kinect Fusion y la opción para codificar en HTML5/JavaScript \cite{kinect_new}.

\section{Requisitos del Sistema}
Esta sección describe los requisitos funcionales del sistema, sus interfaces
externas, las condiciones de excepción y las clases de pruebas que se harán
para verificar que los requisitos se cumplen.
\subsection{Requisitos Funcionales}
Los requisitos funcionales definen el comportamiento del sistema. Es decir,
describen lo que debe hacer el sistema.
\begin{table}[h!]
        \centering
        \begin{tabular}{c l}
                \textbf{RF1} & Obtener la posición de las manos y la cadera \\
                \textbf{RF2} & Obtener la velocidad de la mano derecha \\
                \textbf{RF3} & Calcular tono y volumen en base a la posición
                y velocidad de las manos \\
                \textbf{RF4} & Reproducir sonido en base al tono y volumen \\
                \textbf{RF5} & Proyectar modelo 3D de guitarra \\
        \end{tabular}
        \caption{Requisitos funcionales}
        \label{tab:req_func}
\end{table}

\subsection{Requisitos de Interfaces}
La tabla~\ref{tab:event} muestra la lista de eventos externos a los que el sistema responde.
La primera columna es el nombre del evento; la segunda es la descripción del
mismo. El “iniciador” es la componente externa al sistema que inicia el evento.
Los parámetros son los datos asociados al evento. La respuesta es el nombre de
una respuesta, cuya descripción está en la tabla~\ref{tab:respuesta}.
La tabla~\ref{tab:respuesta} muestra las respuestas del sistema frente a eventos externos.
\begin{table}[h!]
        \centering
        \begin{tabular}{|p{2cm}|p{4cm}|c|p{3cm}|p{25mm}|}
                \hline
                \textbf{Evento} & \textbf{Descripción} & \textbf{Iniciador} &
                \textbf{Parámetros} & \textbf{Respuesta} \\
                \hline
                Recepción de datos de Kinect & Se recibe información del
                Kinect sobre los puntos del esqueleto. & Kinect & Colección
                de puntos 3D del esqueleto del usuario. & Tono y volumen \\
                \hline
                Tocar guitarra & Se reproduce un sonido de guitarra cuando la
                mano entra a un área de reproducción. & Usuario & Velocidad de
                la mano derecha y posición de la mano izquierda & Reproducción
                de sonido \\
                \hline
        \end{tabular}
        \caption{Eventos externos}
        \label{tab:event}
\end{table}

\begin{table}[h!]
        \centering
        \begin{tabular}{|p{3cm}|p{6cm}|p{4cm}|}
                \hline
                \textbf{Respuesta} & \textbf{Descripción} & \textbf{Parámetros} \\
                \hline
                Tono y volumen & Calcula el tono basado en la distancia entre
                los puntos de la mano izquierda y la cadera. Calcula el volumen
                basado en la velocidad de la mano derecha. & Colección de
                puntos 3D del esqueleto del usuario. \\
                \hline
                Reproducción de sonido & Reproduce el sonido correspondiente al
                tono en el que se encuentra el sistema a un volumen basado en
                la velocidad de la mano derecha. & Velocidad de mano derecha. \\
                \hline
        \end{tabular}
        \caption{Respuestas del sistema}
        \label{tab:respuesta}
\end{table}

\section{Requisitos de Ambiente}
Esta sección describe el hardware y software relevante para el sistema Air
Guitar.
\subsection{Requisitos del Ambiente de Desarrollo}
\subsubsection{Hardware de Desarrollo}
\subsubsection{Software de Desarrollo}

\section{Arquitectura del Sistema}
\subsection{Diagrama de Flujo de Datos}

La figura~\ref{fig:flujo} muestra los flujos de datos en el sistema.

\begin{figure}[h!]
        \centering
        \includegraphics[width=0.6\textwidth]{../imagenes/diagrama_de_flujo.png}
        \caption{Diagrama de flujo de datos}
        \label{fig:flujo}
\end{figure}
\newpage

\subsection{Descripción de Módulos}
\subsubsection{Módulo 1(Abreviatura)}
\subsection{Matriz de Requisitos Funcionales y Componentes}

\section{Gestión de Riesgos}
\subsection{Supuestos}
\subsection{Dependencias}
\subsection{Restricciones}
\subsection{Riesgos}
\subsubsection{Riesgo 1}

\newpage 
\appendix 
\newpage

\addcontentsline{toc}{section}{Referencias}
\begin{thebibliography}{99}
\bibitem{depth_range}\url{http://msdn.microsoft.com/en-us/library/hh973078.aspx#Depth_Ranges} \\
        Microsoft Developer Network, 05 Nov 2013
\bibitem{air_guitar}\url{https://en.wikipedia.org/wiki/Air_guitar#Contests} \\
        Wikipedia, 05 Nov 2013
\bibitem{kinect_spec}\url{http://msdn.microsoft.com/en-us/library/jj131033.aspx} \\
        Microsoft Developer Network, 05 Nov 2013
\bibitem{kinect_release}\url{http://en.wikipedia.org/wiki/Kinect#History} \\
        Wikipedia, 05 Nov 2013
\bibitem{sdk}\url{http://msdn.microsoft.com/en-us/library/hh855347.aspx} \\
        Microsoft Developer Network, 05 Nov 2013
\bibitem{kinect_new}\url{http://blogs.msdn.com/b/kinectforwindows/archive/2013/09/16/updated-sdk-with-html5-kinect-fusion-improvements-and-more.aspx} \\
        Kinect for Windows Blog, 05 Noc 2013
\end{thebibliography}

\end{document} 
%%% Local Variables: %%% mode: latex %%% TeX-master: t %%% End:
