\documentclass[a4paper,10pt]{article}
\usepackage[utf8]{inputenc}
\usepackage[spanish]{babel}
\usepackage{graphicx}
\usepackage{epstopdf}
\usepackage{graphicx}
\usepackage{listings}
\usepackage{textcomp}
\usepackage{array}
\usepackage{hyperref}
\usepackage{tabularx}
\usepackage[margin=25mm]{geometry}

\lstset{language=c++}

\title{Documento de Implementación}
\author{Carlos Bergen Dyck \and Gustav Svensk}
\begin{document}

\renewcommand{\arraystretch}{1.5}
\maketitle
\begin{center}
        {\large Versión 0.1}
\end{center}
\newpage
\section{Enumeración de Archivos de Codigo Fuente}
Los archivos de código fuente son:
\begin{itemize}
\item{air\_guitarApp.cpp}
\item{moduleCapture.cpp}
\item{moduleCapture.h}
\item{modulePresentation.cpp}
\item{modulePresentation.h}
\item{moduleProcessing.cpp}
\item{moduleProcessing.h}
\end{itemize}

\section{moduleCapture}
\label{sec:captura}
El módulo con la interfaz al Kinect.
\subsection{Estructuras de Datos}
\begin{itemize}
\item{resolution\_t - Estructura que contiene la resolución de la ventana en pixels}
\item{point - Estructura para representar un punto}
\item{HandsHip - Estructura con los puntos interesante del esqueleto, en metros}
\end{itemize}
\subsection{Clases}
moduleCapture - Clase que representa el módulo.
Tiene métodos para obtener las posiciones de los puntos interesantes del
esqueleto del usuario, para obtener un frame de la cámara de color y para
convertir un punto en espacio de esqueleto (metros) a un punto en el espacio de
la ventana (pixels).

\section{air\_guitarApp}
Archivo donde esta la aplicación principal. Usa la librería cinder.
\subsection{Clases}
AirGuitarApp - aplicación principal.
Tiene métodos para pintar en pantalla (draw), manejar pulses de teclas
(keyDown), actualización (update), inicialización (setup) y eventos del GUI
(guiEvent) Tiene los módulos como atributos para usar sus métodos y atributos.

\section{modulePresentation}
El módulo con la interfaz a la pantalla y altavoces.
\subsection{moduleProcessing}
modulePresentation - Clase que representa el módulo.
Tiene métodos para calcular el tono, calcular el volumen y si una nota esta
tocado ahora.

\end{document} 
%%% Local Variables: %%% mode: latex %%% TeX-master: t %%% End:
