\documentclass[a4paper,10pt]{article}
\usepackage[utf8]{inputenc}
\usepackage[spanish]{babel}
\usepackage{graphicx}
\usepackage{epstopdf}
\usepackage{graphicx}
\usepackage{listings}
\usepackage{textcomp}
\usepackage{array}
\usepackage{hyperref}
\usepackage{tabularx}
\usepackage[margin=25mm]{geometry}

\lstset{language=c++}

\title{Documento de Implementación}
\author{Carlos Bergen Dyck \and Gustav Svensk}
\begin{document}

\renewcommand{\arraystretch}{1.5}
\maketitle
\begin{center}
        {\large Versión 0.1}
\end{center}
\newpage
\section{Enumeración de Archivos de Codigo Fuente}
Los archivos de código fuente son:
\begin{itemize}
\item{air\_guitarApp.cpp}
\item{moduleCapture.cpp}
\item{moduleCapture.h}
\item{modulePresentation.cpp}
\item{modulePresentation.h}
\item{moduleProcessing.cpp}
\item{moduleProcessing.h}
\end{itemize}

\section{moduleCapture}
\label{sec:captura}
El módulo con la interfaz al Kinect.
\subsection{Estructuras de Datos}
\begin{itemize}
\item{resolution\_t - Estructura que contiene la resolución de la ventana en pixels}
\item{point - Estructura para representar un punto}
\item{HandsHip - Estructura con los puntos interesante del esqueleto, en metros}
\end{itemize}
\subsection{Clases}
moduleCapture - Clase que representa el módulo.
Hay métodos para obtener las posiciones de los puntos interesantes del esqueleto del 
usuario, para obtener un frame de la camara de color y para convertir un punto en espacio de
esqueleto (metros) a un punto en el espacio de la ventana (pixels).

% \section{Módulo de Procesamiento}
% \label{sec:procesamiento}
% \subsection{Definición de Módulo}
% \begin{tabularx}{\textwidth}{p{25mm} X}
%         \textbf{Propósito} & Calcular el tono y volumen y ver si una nota fue tocada.\\
%         \textbf{Alcance} & Calcula el tono basado en la distancia de la mano izquierda y la cadera. Calcula el volumen basado en la velocidad de la mano derecha y ve si una nota fue tocada si la mano derecha se encuentra sobre un área predefinida.\\
%         \textbf{Dependencias} & Depende del modulo de captura.\\
%         \textbf{Supuestos} & Un funcionamiento correcto del modulo de captura.\\
%         \textbf{Restricciones} & Existira un rango de tonos disponibles para tocar y tendra un limite de volumen.\\
%         \textbf{Estructura General} & Recibe los puntos obtenidos del modulo de captura. Entrega el tono y volumen de la nota actual y si esta debe ser reproducida. Entrega la posición donde debe ser proyectada la guitarra. \\
% \end{tabularx}
% \subsection{Declaraciones Públicas}
% Esta sección enumera constantes, tipos y variables del módulo, visibles para
% otros módulos.
% \subsubsection{Constantes Públicas}
% \begin{tabular}{| p{30mm} | p{10cm} |}
%         \hline
%         \textbf{Nombre de la \mbox{constante}} & \textbf{Descripción} \\
%         \hline
%         E & Distancia de la nota E\\
%         F & Distancia de la nota F\\
%         F\# & Distancia de la nota F\#\\
%         G & Distancia de la nota G\\
%         G\# & Distancia de la nota G\#\\
%         A & Distancia de la nota A\\
%         A\# & Distancia de la nota A\#\\
%         B & Distancia de la nota B\\
%         C & Distancia de la nota C\\
%         C\# & Distancia de la nota C\#\\
%         D & Distancia de la nota D\\
%         D\# & Distancia de la nota D\#\\
%          boxSize & Tamaño del area donde se puede tocar la nota\#\\
%         \hline
% \end{tabular}

                

% \subsubsection{Variables Públicos}
% \begin{tabular}{| p{30mm} | p{10cm} |}
%         \hline
%         \textbf{Nombre de la \mbox{variable}} & \textbf{Descripción} \\
%         \hline
%         Tono & El tono de la nota.\\
%         Volumen & El volumen de la nota.\\
%         \hline
% \end{tabular}
% \subsection{Funciones Públicas}
% Las siguientes funciones son accesibles desde otros módulos. Otros módulos
% tienen acceso a la funcionalidad de este módulo mediante estas funciones.
% ~\\

% \begin{tabular}{| p{30mm} | p{10cm} |}
%         \hline
%         \textbf{Nombre de la \mbox{función}} & \textbf{Descripción breve} \\
%         \hline
%         calculateTone & Determina el tono de la nota que se va a tocar.\\
%         calculateVolume & Determina el volumen de la nota que se va a tocar.\\
%         playedNote & Determina si el usuario toco una nota.\\
%         \hline
% \end{tabular}
% \subsection{Estructuras de Datos}
% \subsection{Diseño Detallado de las Funciones}
% \subsubsection{calculateTone}
% \begin{tabularx}{\textwidth}{p{25mm} X}
%         \textbf{Descripción} & Obtener el tono de la nota actual. \\
%         \textbf{Dependencias} & \\
%         \textbf{Prototipo} & \lstinline{void calculateTone(HandsHip *handship);}\\
%         \textbf{Parámetro} & \textbf{Explicación} \\
%         \begin{tabular}{p{2cm} l}
%                 handsHip & Posición de las manos y la cadera \\
%         \end{tabular}\\

%         \textbf{Retorno} & \\
%         \textbf{Proceso} & 
%                \begin{lstlisting}[breaklines=true]^^J
%             chordHand = handship->chordHand;^^J
%             hip = handship->hip;^^J
%             float distance = (chordHand.X - hip.X)*(chordHand.X - hip.X) + (chordHand.Y - hip.Y)*(chordHand.Y - hip.Y);^^J
%             if(distance > E)\{^^J
%             \    note = 0;^^J
%             \}else if(distance > F)\{^^J
%             \    note = 1;^^J
%             \}else if(distance > F#)\{^^J
%             \    note = 2;^^J
%             \}else if(distance > G)\{^^J
%             \    note = 3;^^J
%             \}else if(distance > G#)\{^^J
%             \    note = 4;^^J
%             \}else if(distance > A)\{^^J
%             \    note = 5;^^J
%             \}else if(distance > A#)\{^^J
%             \    note = 6;^^J
%             \}else if(distance > B)\{^^J
%             \    note = 7;^^J
%             \}else if(distance > C)\{^^J
%             \    note = 8;^^J
%             \}else if(distance > C#)\{^^J
%             \    note = 9;^^J
%             \}else if(distance > D)\{^^J
%             \    note = 10;^^J
%             \}else\{^^J
%             \    note = 11;^^J
%             \}^^J
%         \end{lstlisting} 
%  \\
% \end{tabularx}

% \subsubsection{Función calculateVolume}
% \begin{tabularx}{\textwidth}{p{25mm} X}
%         \textbf{Propósito} & Calcular el volumen de la nota\\
%         \textbf{Dependencias} & \\
%         \textbf{Prototipo} & \lstinline{void calculateVolume(HandsHip *handsHip);}\\
%         \textbf{Parámetro} & \textbf{Explicación} \\
%         \begin{tabular}{p{2cm} l}
%                 handsHip & La posición actual de las manos y la cadera \\
%         \end{tabular} \\

%         \textbf{Retorno} &\\
%         \textbf{Proceso} & 
%         \begin{lstlisting}[breaklines=true]^^J
%             rightHand = handsHip->rightHand;^^J
%         float distance = (rightHand.X - lastRightHand.X)*(rightHand.X - lastRightHand.X)^^J
%                         + (rightHand.Y - lastRightHand.Y)*(rightHand.Y - lastRightHand.Y);^^J
%         if(distance>maxDistance)\{^^J
%         \    volume = maxDistance;^^J
%         \}else\{^^J
%          \    volume = distance/maxDistance;^^J
%         \}^^J
%         lastRightHand = rightHand;^^J
%         \end{lstlisting}
%              \\
% \end{tabularx}

% \subsubsection{Función playedNote}
% \begin{tabularx}{\textwidth}{p{25mm} X}
%         \textbf{Propósito} & Determina si el usuario toco una nota\\
%         \textbf{Dependencias} & \\
%         \textbf{Prototipo} & \lstinline{boolean playedNote(HandsHip *handsHip);}\\
%         \textbf{Parámetro} & \textbf{Explicación} \\
%         \begin{tabular}{p{2cm} l}
%                 handsHip & La posición actual de las manos y la cadera \\
%         \end{tabular} \\

%         \textbf{Retorno} & Valor booleano que indica si una nota fue tocada\\
%         \textbf{Proceso} & 
%         \begin{lstlisting}[breaklines=true]^^J
%             hip = handsHip->hip;^^J
%             rightHand = handsHip->rightHand;^^J
%             ax = hip.X-boxSize;^^J
%             ay = hip.Y-boxSize;^^J
%             bx = hip.X+boxSize;^^J
%             by = hip.Y-boxSize;^^J
%             dx = hip.X+boxSize;^^J
%             dy = hip.Y+boxSize;^^J
%             bax = bx - ax;^^J
%             bay = by - ay;^^J
%             dax = dx - ax;^^J
%             day = dy - ay;^^J
%             if ((x-ax)*bax+(y-ay)*bay<0.0) isOnPlayingArea = false;^^J
%             if ((x-bx)*bax+(y-by)*bay<0.0) isOnPlayingArea = false;^^J
%             if ((x-ax)*dax+(y-ay)*day<0.0) isOnPlayingArea = false;^^J
%             if ((x-ax)*bax+(y-ay)*bay<0.0) isOnPlayingArea = false;^^J
%             if ((x-ax)*bax+(y-ay)*bay<0.0) isOnPlayingArea = false;^^J
%             if ((x-bx)*bax+(y-by)*bay<0.0) isOnPlayingArea = false;^^J
%             if ((x-ax)*dax+(y-ay)*day<0.0) isOnPlayingArea = false;^^J
%             if ((x-ax)*bax+(y-ay)*bay<0.0) isOnPlayingArea = false;^^J
%             if(isOnPlayingArea \&\& !notePlayed)\{^^J
%             \    notePlayed = true;^^J
%             \    return = true;^^J
%             \}^^J
%             if(!isOnPlayingArea) notePlayed = false;^^J
%             return false;^^J
%         \end{lstlisting}
%              \\
% \end{tabularx}

% \section{Módulo de Presentación}
% \label{sec:presentacion}
% \subsection{Definición de Módulo}
% \begin{tabularx}{\textwidth}{p{25mm} X}
%         \textbf{Propósito} & Se encarga de dibujar las pantallas de la aplicación y la guitarra así como reproducir los sonidos de las notas tocadas.\\
%         \textbf{Alcance} & Reproduce el sonido correspondiente a la nota calculada y proyecta un modelo 3D de una guitarra sobre el usuario.\\
%         \textbf{Dependencias} & Depende del modulo de procesamiento y captura.\\
%         \textbf{Supuestos} & Tener una tarjeta gráfica apropiada así como una tarjeta de sonido.\\
%         \textbf{Restricciones} & Reproducir los sonidos correspondientes definidos anteriormente.\\
%         \textbf{Estructura General} & Recibe la posición de la guitarra, el
%         tono y volumen de la nota y si esta fue tocada. Dibuja en la pantalla
%         la imagen del usuario con un modelo 3D de una guitarra proyectada y
%         reproduce el sonido apropiado.\\
% \end{tabularx}
% \subsection{Declaraciones Públicas}
% Esta sección enumera constantes, tipos y variables del módulo, visibles para
% otros módulos.
% \subsubsection{Estructuras de Datos}
% \begin{tabular}{| p{30mm} | p{10cm} |}
%         \hline
%         \textbf{Nombre de la \mbox{estructura}} & \textbf{Descripción} \\
%         \hline
%         Guitar & Objeto que contiene la información de la guitarra, la imagen, que tipo es, el objeto 3d con orientación y escala\\
%         \hline
% \end{tabular}
% \subsection{Funciones Públicas}
% Las siguientes funciones son accesibles desde otros módulos. Otros módulos
% tienen acceso a la funcionalidad de este módulo mediante estas funciones.~\\

% \begin{tabular}{| p{30mm} | p{10cm} |}
%         \hline
%         \textbf{Nombre de la \mbox{función}} & \textbf{Descripción breve} \\
%         \hline
%         drawImage & Pinta la imagen del Kinect al fondo \\
%         \hline
%         drawGuitar & Pinta la guitarra sobre el usuario \\
%         \hline
%         playTone & Toca un tono \\
%         \hline
% \end{tabular}

% \subsubsection{Variables Privadas}
% \begin{tabular}{| p{30mm} | p{10cm} |}
%         \hline
%         \textbf{Nombre de la \mbox{variable}} & \textbf{Descripción} \\
%         \hline
%         guitar & el objeto que contiene la información de la guitarra\\
%         \hline
% \end{tabular}
% \subsection{Funciones Privadas}
% Las siguientes funciones auxiliares son privadas de este módulo; otros módulos
% no las pueden usar.~\\

% \begin{tabular}{| p{30mm} | p{10cm} |}
%         \hline
%         \textbf{Nombre de la \mbox{función}} & \textbf{Descripción breve} \\
%         \hline
%         calcGuitarPos & Calcula la posición normalizada de la guitarra en base a la cadera y longitud de guitarra \\
%         \hline
% \end{tabular}
% \subsection{Diseño Detallado de las Funciones}
% \subsubsection{drawImage}
% \begin{tabularx}{\textwidth}{p{25mm} X}
%         \textbf{Propósito} & Pinta la imagen obtenido del Kinect al fondo\\
%         \textbf{Dependencias} & openFrameworks \\
%         \textbf{Prototipo} & \lstinline{void drawImage(ImageFrame *image);}\\
%         \textbf{Parámetro} & \textbf{Explicación} \\
%         \begin{tabular}{p{2cm} l}
%                image & La imagen a pintar, obtenido del modulo de captura\\
%         \end{tabular}\\

%         \textbf{Retorno} & Nada\\
%         \textbf{Proceso} & 
%         \begin{lstlisting}[breaklines=true]^^J
%         ofImage tmpImage = ofImage(image);^^J
%                 tmpImage.draw(0,0);^^J
%         \end{lstlisting}

% \end{tabularx}
% \subsubsection{drawGuitar}
% \begin{tabularx}{\textwidth}{p{25mm} X}
%         \textbf{Propósito} & Pinta la guitarra sobre la imagen del usuario\\
%         \textbf{Dependencias} & openFrameworks y calcGuitarPos\\
%         \textbf{Prototipo} & void drawGuitar()\\
%         \textbf{Retorno} & Nada \\
%         \textbf{Proceso} & 
%         \begin{lstlisting}[breaklines=true]^^J
%                 int positions[4];^^J
%                 positions = calculateGuitarPosition();^^J
%                 guitar.setRotation(atan2(position[3] - position[1], position[2] - position[0]));^^J
%                 guitar.draw(position[0], position[1]);^^J
%         \end{lstlisting}
% \end{tabularx}
% \subsubsection{playTone}
% \begin{tabularx}{\textwidth}{p{25mm} X}
%         \textbf{Propósito} & Toca el tono recibido como parametro \\
%         \textbf{Dependencias} & Módulo de Procesamiento y openFrameworks\\
%         \textbf{Prototipo} & void playTone(int tone, int volume);\\
%         \textbf{Parámetro} & \textbf{Explicación} \\
%         \begin{tabular}{p{2cm} l}
%                 tone & El tono a tocar \\
%                 volume & El volumen debido a la velocidad de la mano derecha\\
%         \end{tabular}\\
%         \textbf{Retorno} & Nada \\
%         \textbf{Proceso} & 
%         \begin{lstlisting}[breaklines=true]^^J
%                 int total_volume = volume * ui->volume;^^J
%                 ofSoundPlayer note;^^J
%                 ofSoundSetVolume(volume);
%                 note.load(tone);^^J
%                 note.play();^^J
%         \end{lstlisting}

%         \\
% \end{tabularx}

% \newpage

% \subsubsection{calcGuitarPos}
% \begin{tabularx}{\textwidth}{p{25mm} X}
%         \textbf{Propósito} & Calcula la posición de la base y cabeza de la guitarra\\
%         \textbf{Dependencias} & Módulo de Captura \\
%         \textbf{Prototipo} & \lstinline{int* calcGuitarPos(HandsHip handsHip);}\\
%         \textbf{Parámetro} & \textbf{Explicación} \\
%         \begin{tabular}{p{2cm} l}
%                 handsHip & Información sobre las posiciones de las manos y cadera del usuario \\
%         \end{tabular}\\

%         \textbf{Retorno} & Un arreglo de enteros que representa las posiciones de la base y cabeza de la guitarra\\
%         \textbf{Proceso} & 
%         \begin{lstlisting}[breaklines=true]^^J
%                 int pos* = new int[4];^^J
%                 int hipx = handsHip->hipPosition[0];^^J
%                 int hipy = handsHip->hipPosition[1];^^J
%                 int handx = handsHip->chordHandPosition[0];^^J
%                 int handy = handsHip->chordHandPosition[1];^^J
%                 float dx = handx - hipx;^^J
%                 float dy = handy - hipy;^^J
%                 float normValue sqrt(dx*dx + dy*dy);^^J
%                 dx = dx/normValue;^^J
%                 dy = dy/normValue;^^J
%                 pos[0] = hipx;^^J
%                 pos[1] = hipy;^^J
%                 pos[2] = (int)hipx + dx*gitarr->length;^^J
%                 pos[3] = (int)hipy + dy*gitarr->length;^^J
%                 return pos;^^J
%         \end{lstlisting}

\end{document} 
%%% Local Variables: %%% mode: latex %%% TeX-master: t %%% End:
